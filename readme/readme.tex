\documentclass[letterpaper,12pt]{article}

\usepackage{amsfonts}
\usepackage{amsmath}
\usepackage{geometry}
\usepackage{setspace}
\usepackage{booktabs}
\usepackage{hyperref}
\usepackage{xcolor}
\usepackage{harvard}
\usepackage{graphicx}
\usepackage[capposition=top]{floatrow}
\usepackage[small,bf]{caption}
%\usepackage{fixltx2e}
\usepackage[flushleft]{threeparttable}
\usepackage{rotating}
\usepackage{multirow}
\usepackage{float}

\geometry{top=1in,bottom=1in,left=1in,right=1in}

\begin{document}

\section{Overview}

This readme file describes how to set up and run the MATLAB programs used to produce the quantitative results in section IV of the paper.
The following programs and data are included with this file:
\begin{itemize}
	\item[1.] MATLAB scripts and a C program to compute the excess demand along different quantiles of the housing supply distribution. This code is pre-configured to be run for all experiments in the paper at those service flow and price functions that minimize the difference between the supply and demand distribution of housing quality (i.e. the market-clearing functions).
	\item[2.] The data inputs for the programs from 1. These include household level data from Census and ACS, the estimated supply quality distributions, and several other inputs that were obtained from different sources and are listed in the part III.B of the paper.
	\item[3.] MATLAB scripts used to generate the graphs and the statistics in section IV of the paper, using as input the results of the computational procedures from part 1.
\end{itemize}


\section{Installation}

Unpack the contents of the archive into an empty directory. To run the experiments, you need MATLAB version 2014a or more recent. Most likely, you will have to re-compile the C program that is called from MATLAB via the mex interface. While the archive contains a binary file, it might be incompatible with your version of Windows/MATLAB etc. To re-compile, start MATLAB and set the working directory to the root of the unpacked archive. Then execute the command\\
\verb|mex hdemand_grid.c| ,\\
which should run successfully if you have configured a C compiler to be used with MATLAB. If you have not yet configured a C compiler to be used with MATLAB for compilation of MEX functions, please refer to the MATLAB documentation to perform this step. On Windows, the latest version of the Microsoft C++ runtime works fine and is available for free. The pre-compiled binary was created with MATLAB 2014a on Windows 7 (64 bit) using MS Visual Studio 2012.

\section{Contents and Structure of the Code}

The root directory contains the MATLAB programs required to run the experiments. The following list gives a brief description of relevant individual script files:
\begin{itemize}
	\item \verb|main_control.m|\\
		  This is the master script to run computational experiments. It contains a section at the top where you can configure the file with experiment definitions, the name of the experiment and several other settings. See the comments in the code for details.
	\item \verb|exper_def.m|\\
		  Contains experiment definitions. This includes the guess vectors for price and/or service flow functions, the expectations scenario, and parameters that vary between experiments. The experiments defined in the script correspond to the ones presented in the paper.
	\item \verb|set_parameters.m|\\
		  Called by \verb|main_control.m| once in the beginning to set the values of many model parameters.
	\item \verb|load_surveys.m|\\
		  Called by \verb|main_control.m| once in the beginning to load the survey data files.
	\item \verb|prep_current_survey.m|\\
	      Called by \verb|main_control.m| before computing an experiment to set parameters that depend on the specific experiment, such as configuration of the grids for the state variables.
	\item \verb|calc_ex_dem_interp.m|\\
		  Function that computes the household dynamic program once (for all age groups), and then uses the resulting policy functions to calculate the model-predicted choices of all households in the Census/ACS survey. Then computes excess demand at quantiles of the quality supply distribution.
	\item \verb|hdemand.m|\\
		  Computes on the value function and policy functions of the household dynamic program for one year/age group. Needs the continuation value function as input. Runs a parallel for-loop over the cash dimension of the state space.
	\item \verb|hdemand_grid.c|\\
	      C program to be compiled for MATLAB's mex interface. Performs the optimization of the household program for one period given parameters and continuation values on the grid of state variables.		  
\end{itemize}

The data files needed for the calculation of excess demand in housing quality are contained in the \verb|data| subfolder. They are
\begin{itemize}
	\item \verb|census_demand.txt| and \verb|acs05_demand.txt|\\
		  Census and ACS micro data with household cash imputed as described in the appendix of the paper. Used to compute demand for housing quality by movers.
	\item \verb|censusdata2mn_21.mat| and \verb|newacsdata2mn_21.mat|\\		  
		  MATLAB data files containing spline objects that represent the supply distribution of housing quality. Constructed as described in the paper.
\end{itemize}


The \verb|results| subfolder contains an output file \verb|experlist.mat| that already consists of the equilibrium value, policy, service flow and price functions for each experiment computed in the paper. Note that this file also contains the optimal policies for each household in the census/ACS at the equilibrium prices, since these data are required to generate the cross-sectional results presented in the paper.\\

The \verb|results| subfolder further contains
\begin{itemize}
	\item two MATLAB programs, \verb|make_graphs_exper.m| and \verb|cross_section.m|, that can be used to create the graphs and cross-sectional statistics of the section IV in the paper, respectively.
	\item several other MATLAB script files that are called by  \verb|make_graphs_exper.m| and \verb|cross_section.m|.
	\item a subfolder \verb|logs| containing output log files from the computational runs of the individual experiments. The logs files show the distance between quality supply and demand distributions at different quantiles and at the service flow or price function for which this distance was minimized. 
\end{itemize}

\section{Running Experiments}

To run an experiment, simply set the variable \verb|exper_type| at the top of \verb|main_control.m| to the name of the file containing the experiment's name (see \verb|exper_def.m| for the names of the already defined experiments). Then execute \verb|main_control.m|, with the working directory being set to the root folder of the unpacked archive. 

Another important setting before running the experiment is whether to enable parallelization of the computation procedure. If you have MATLAB's Parallel Computing Toolbox installed, you can set the number of workers to any feasible number on your machine. The code will then create a local parallel pool containing that number of workers and run the household dynamic program using a parallel for-loop.

If you do not have the Parallel Computing Toolbox or you do not want to run the code in parallel mode, simply set \verb|no_par_processes| to 1.

Also note that in order to run the code, you need a binary file for \verb|hdemand_grid.c| that is compatible with your platform. You can try to run it with the binary already contained in the archive; however, if this leads to an error message, you will need to recompile using the mex command mentioned above under ``Installation''.

\section{Producing the Paper's Graphs and Statistics}

The archive also contains two MATLAB programs (\verb|make_graphs_exper.m| and\\
 \verb|cross_section.m|) that can be used to produce the graphs in figures 6, 7 and 9 of the paper, and the cross-sectional statistics of tables 3 and 4. These programs require as input a MATLAB data file consisting of a structure with the output of all experiments defined in \verb|exper_def.m|. As mentioned above, the results folder already contains such a file (\verb|experlist.m|); however, you could modify this file by adding the results of other experiments you run. The programs also access some of the data files and should be self-explanatory.



\end{document}